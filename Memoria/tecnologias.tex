\section{Tecnologías utilizadas}
Las tecnologías utilizadas han sido las siguientes:

\subsection{C++}
Aunque se comenzó utilizando C en las primeras versiones del TFG, finalmente se optó por utilizar C++. Esto fue debido a que el cambio a C++ permitió mantener el trabajo ya realizado en C, y posee estructuras de datos ya implementadas como mapas o vectores que reducen en gran medida el tiempo de programación, además de permitir la orientación a objetos.

\subsection{MySQL}
Al igual que en la versión anterior del proyecto, la BBDD que usamos será la relacional que realizó Diego Ruiz Aguado en el 2012.

\subsection{Scripting}
Se crearon pequeños scripts que se encargan de importar la BBDD, exportar las tablas en el formato necesario y compilar el proyecto

\subsection{HTML y JavaScript}
Para la representación gráfica del problema se ha utilizado JS para leer y parsear la información almacenada en un fichero y HTML para su visualización (para la creación del grafo se ha utilizado la librería \href{http://visjs.org/docs/network/}{vis.js}).

\subsection{Git}
Se ha utilizado Git como sistema de versiones, y todo el código se puede descargar y consultar en  \href{https://github.com/cavasanchez/TFG}{Github}



