\documentclass[12pt,a4paper]{article}
\usepackage[latin1]{inputenc}
\usepackage{amsmath}
\usepackage{amsfonts}
\usepackage{amssymb}
\usepackage{graphicx}
\usepackage{hyperref}

\begin{document}
	
	\section{Importar proyecto}
	A continuaci�n se enumeran los pasos necesarios para importar el proyecto a ecplise. Si se utilizara un editor diferente (Netbeans, Xcode), los pasos ser�an an�logos
	\begin{enumerate}
		\item Si no se dispusiera del c�digo, descargarlo desde aqu� \href{https://github.com/cavasanchez/TFG}{https://github.com/cavasanchez/TFG} .
		\item Para importar el proyecto basta con Import $\rightarrow$ Import from workspace y seleccionar la carpeta TFG/TGG-cpp-master.
		\item Compilar el proyecto y comprobar que no existen errores. En caso de error comprobar los siguientes par�metros de configuraci�n:
		\begin{itemize}
			\item El compilador es GNU make builder.
			\item Toolchain es Cross GCC.
		\end{itemize}
	\end{enumerate}
	
	\section{Base de datos}
	Los 3 modelos de BBDD se pueden encontrar en la carpeta "Originales/BBDD normalizadas". Para importarlas al MySQL workbench basta con crear un nuevo schema e importar el .mysql desde la opci�n "import from self-contained file".
	
	
	\section{Par�metros y configuraci�n del problema}
	
	\subsection{Elegir el modelo para el problema}
	De los 3 modelos de los que disponemos, tenemos de cada uno de ellos una versi�n del problema con 20, 100 o todos los vuelos. Para cambiar entre uno y otro, basta con editar el par�metro $RESORCES_FOLDER$ que tenemos en el fichero constants.h. Para elegir la base de datos requierida escribir la ruta de la forma \textit{./Resources/BLO2-20flights/}\textit{./Resources/BLO1-100flights/} . La parte de BLO-XXX indica la base de datos, y la parte -XXXflights indica el n�mero de vuelos. A nivel interno lo que se encarga es de abrir la carpeta adecuada donde se encuentran todos los modelos.
	
	\subsection{Par�metros del heur�stico}
	Tal y como se indicaba en la memoria, el algoritmo desarrollado depende de dos par�metros, ambos se pueden cambiar desde el fichero constants.h . El par�metro N se modifica mediante el par�metro \textit{$NUM\_SOULUTIONS\_TO\_EXAMINE$}, y G mediante \textit{$MAX\_NUMBER\_QUEUE$}

	
	\subsection{N�mero de simulaciones que ejecuta el algoritmo}
	 Par�metro que indica el n�mero de veces que relanzamos el multistart. Se puede modificar en Constants.h en la variable \textit{$MAX\_ITERATIONS$}

	\subsection{N�mero de veces que se ejecuta el algoritmo}
	Indica el n�mero de veces que se reinicia el problema. Se puede modificar en Constants.h en la variable \textit{$NUM\_SIMULATIONS$}
	
	
	\section{Contacto}
	Para cualquier duda o consulta, no dudar en escribir a cavasanchez@gmail.com

\end{document}