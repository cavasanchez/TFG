\chapter{Resultados experimentales}

\section{Comparación de parámetros $P$ y $N$}
\section{Evolución de la función objetivo}
\section{Pruebas realizadas}



A continuación se detallan los resultados obtenidos en distintos problemas. Los resultados corresponden a la media de 100 simulaciones con 100 iteraciones cada una. En estos datos además con la hipótesis más estricta de que la capacidad de un sector en un instante $t$ es 1

\section{Base de datos 1}
\subsection{20 vuelos}
\textcolor{red}{Sobre el 90\%-100\%}
\subsection{100 vuelos}
\textcolor{red}{Sobre el 60\%-70\%}
\subsection{812 vuelos}
\textcolor{red}{SIN PROBAR}

\section{Base de datos 2}
\subsection{20 vuelos}
\textcolor{red}{Sobre el 95\%-100\%}
\subsection{100 vuelos}
\textcolor{red}{Sobre el 90\%}
\subsection{3196 vuelos}
\textcolor{red}{SIN PROBAR}

\section{Base de datos 3}
\subsection{20 vuelos}
\textcolor{red}{Sobre el 95\%-100\%}
\subsection{100 vuelos}
\textcolor{red}{Sobre el 80\%}
\subsection{6475 vuelos}
\textcolor{red}{SIN PROBAR}

\begin{tikzpicture}[->,>=stealth',shorten >=1pt,auto,node distance=3cm,
thick,main node/.style={circle,draw,font=\sffamily\Large\bfseries}]

\node[main node] (1) {1};
\node[main node] (2) [below left of=1] {2};
\node[main node] (3) [below right of=2] {3};
\node[main node] (4) [below right of=1] {4};

\path[every node/.style={font=\sffamily\small}]
(1) edge node [left] {0.6} (4)
edge [bend right] node[left] {0.3} (2)
edge [loop above] node {0.1} (1)
(2) edge node [right] {0.4} (1)
edge node {0.3} (4)
edge [loop left] node {0.4} (2)
edge [bend right] node[left] {0.1} (3)
(3) edge node [right] {0.8} (2)
edge [bend right] node[right] {0.2} (4)
(4) edge node [left] {0.2} (3)
edge [loop right] node {0.6} (4)
edge [bend right] node[right] {0.2} (1);
\end{tikzpicture}



\begin{table}[htbp]
	\centering
	\caption{Comparativa de parámetros $G$ y $N$.}
	\label{fig: comparativa de parámetros $G$ y $N$.}
	\begin{tabular}{|c|l|l|l|l|l|l|l|l|l|}
		\hline
		\textbf{} & \textbf{N=2, G=10} & \textbf{N=2, G=30} & \textbf{N=2, G=50} & \textbf{N=5, G=10} & \textbf{N=5, G=30} & \textbf{N=5, G=50} & \textbf{N=10, G=10} & \textbf{N=10, G=30} & \textbf{N=10, G=50} \\ \hline
		\textbf{Problema 1
			100 vuelos} & \% éxito: 1 &  &  &  &  &  &  &  &  \\ \hline
		\textbf{Problema 2
			100 vuelos} &  &  &  &  &  &  &  &  &  \\ \hline
		\textbf{Problema 3
			100 vuelos} &  &  &  &  &  &  &  &  &  \\ \hline
	\end{tabular}
	
\end{table}



\begin{table}[]
	\centering
	\caption{My caption}
	\label{my-label}
	\begin{tabular}{lllllllllllll}
		\hline
		&  N=10, G=10 & N=10, G=30 & N=10, G=50 & N=20, G=10 & N=20, G=30 & N=20, G=50 \\ \hline
		
		Problema 1 & \multicolumn{1}{c}{\begin{tabular}[c]{@{}c@{}}Máx: 1\\ Med: 2\\ Desv: 3\end{tabular}} &           &           &           &           &           &            &            &            &            &            &            \\
		
		Problema 2 &                                                                                       &           &           &           &           &           &            &            &            &            &            &            \\
		
		Problema 3 &                                                                                       &           &           &           &           &           &            &            &            &            &            &            \\ \hline
	\end{tabular}
\end{table}
Como se puede comprobar, esta combinación de parámetros es la que aporta más aleatoriedad en los resultados, pero es también de media la que aporta mejores resultados, lo que indica que se las malas soluciones son descartadas rápidamente pero se permite explorar un máximo local.


