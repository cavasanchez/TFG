\chapter{Conclusiones}
\section{Valoración de los resultados}
Al final del Trabajo de Fin de Grado se han cumpido todos los objetivos que se establecieron en su momento:
\begin{itemize}
	\item Modelar de nuevo el problema sin que éste dependiera de CIPLEX.
	\item Mejorar el sistema de lectura de datos.
	\item Mejora del heurístico para lanzar los vuelos.
	\item Creación de una simple representación visual del problema
\end{itemize}
Aunque se han cumplido todos los objetivos cumplidos, podemos comparar los resultados que se obtuvieron en la versión anterior del problema con los de este TFG:\\
En la versión de junio de 2012 con una base de datos de 65 vuelos y las capacidades de los sectores a 1, obtuvieron un rcentaje de vuelos cancelados de 11\%. Con el nuevo algoritmo lo reducimos a un 3\%
\section{Futuras líneas de trabajo}
Para enriquecer más el modelo, se podrían añadir algunas de las restricciones que se eliminaron para simplificar el modelo. La más importante sería la de añadir el coste de cancelación de un vuelo, de forma que a la hora de elegir entre 2 vuelos parejos se tenga en cuenta el coste de cancelación


