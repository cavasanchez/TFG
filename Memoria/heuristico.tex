\section{Heurístico}
El algoritmo que se ha creado para el problema está inspirado en el metahurístico constructivo \textit{GRASP (Greedy Randomized Adaptive Search Procedure)} se resume en el siguiente diagrama:

\textcolor{red}{->IMAGEN DEL ALGORITMO EN UML<-}


Los pasos son:
\begin{enumerate}
	
	\item \textbf{Se intentan colocar vuelos con su mejor solución:} se lanzan todos los vuelos de manera aleatoria sin permitir retrasos o desvíos, solo la solución inicial.
	
	\item \textbf{Intercambio de vuelos: } se intenta sustituir uno de los vuelos exitosos del paso anterior por 2 o más vuelos aleatorios a los que no se les halló solución. Para ello:
	
	\item \textbf{Se intentan colocar vuelos permitiendo retrasos}: se lanzan los vuelos que aun no tienen solución, permitiendo retrasos en sus rutas, pero no desvíos.
	
	\item \textbf{Se intentan colocar vuelos permitiendo retrasos y desvíos}: se lanzan los vuelos que aun no tienen solución, permitiendo retrasos y desvíos en sus rutas.
	
	\item \textbf{Se intentan colocar vuelos permitiendo retrasos}: se lanzan los vuelos que aun no tienen solución, permitiendo retrasos en sus rutas, pero no desvíos. En este punto si un vuelo tiene alguna solución factible, se le asignará.
	
	\item \textbf{Se buscan los waypoints sin usar y se les asigna una ruta}: se localizan los waypoints por los que no pasa ningún vuelo a lo largo de todo el problema. Si algún vuelo tiene alguna solución que utilice alguno de estos waypoints, y es factible, se la asigna
	
	\item \textbf{Retrasar vuelos con solución para colocar 2 o más cancelados}: se intenta retrasar alguno de los vuelos con solución factible para poder encontrar de forma aleatoria uno o más vuelos que estaban cancelados.
	
\end{enumerate}

Este esquema se repite el número de iteraciones indicadas. para que no sea un proceso totalmente aleatorio, se han añadido 2 técnicas utilizadas en los algoritmos metahurísticos: La \textit{búqueda tabú} y el \textit{arranque multistart}