\section{Versiones anteriores del problema y nuevos objetivos}

\subsection{Versión anterior}
Este Trabajo Final de Grado es la continuación del trabajo que llevaron a cabo Diego Ruiz Aguado y Gonzalo Quevedo García en 2012 en sus Proyectos Finales de Carrera, los cuales se apoyaron a su vez en la Tesis Doctoral de Alba Agustín  Martín.\\

A continuación se da una breve descripción del trabajo de  Diego Ruiz Aguado y Gonzalo Quevedo García:
\begin{enumerate}
	\item Se mejoró la BBDD que contenía toda la información del problema, pasando de un modelo no relacional y con redundancias a uno relacional y bien estructurado.
	\item Para obtener los datos que necesitaba el problema, se realizó un programa en JAVA que se conectaba a la BBDD y creaba varios ficheros .txt en la que se volcaba toda la información necesaria para el posterior modelado del problema.
	\item A continuación, se leían estos ficheros .txt y se creaban las estructuras de datos necesarias (árbol de rutas, vuelos, wapoints, etc).
	\item Posteriormente una subrutina en C se encargaba de definir un problema de CIPLEX con la función objetivo y las restricciones necesarias.
	\item Finalmente se obtenía la mejor solución del problema.
\end{enumerate}


\subsection{Nuevos objetivos}
La versión anterior del problema adolecía de un importante inconveniente: no podía salir de los máximos locales, ya que el heurístico que utilizaba para lanzar los vuelos era un algoritmo voraz. Por tanto los objetivos marcados para este TFG han sido los siguientes (ordenados en decreciente prioridad):
\begin{enumerate}
	\item \textbf{Mejorar heurístico: }utilizar heurísticos más elaborados que mejoren la solución del problema.
	\item \textbf{Desacoplar el programa de CIPLEX: }con la implementación de los nuevos heurísticos no es necesaria la librería de optimización. Se pasará de un sistema clásico de optimización (función objetivo y restricciones) a una estructura de objetos que permitan un manejo óptimo de las esrteucturas de datos durante el heurístico.
	\item \textbf{Mejorar el sistema de lectura de datos: }el sistema actual crea ficheros .txt que pueden superar las 100.000 lineas. Hay que mejorar este sistema.
	\item \textbf{Representación gráfica: }aunque no es estrictamente necesaria, si se dispusiera de tiempo suficiente se añadiría una representación gráfica de la solución del problema.
\end{enumerate}