\section{Versiones anteriores del problema y nuevos objetivos}

\subsection{Versiones anteriores}
Este Trabajo Final de Grado es la continuación del realizado por los Proyectos Finales de Carrera de Diego Ruiz Aguado y Gonzalo Quevedo García en 2012, los cuales se apoyaron a su vez en la Tesis Doctoral de Alba Agustín  Martín.\\


A continuación se da una breve descipción del trabajo de  Diego Ruiz Aguado y Gonzalo Quevedo García:
\begin{enumerate}
	\item Se mejoró la BBDD que contenía toda la información del problema, pasando de un modelo no relacional y con redundancias a uno relacional y bien estructurado.
	\item Para obtener los datos que necesitaba el problema, se realizó un programa en JAVA que se conectaba a la BBDD y creaba varios ficheros .txt en la que se volcaba información necesaria.
	\item A continuación, se leían estos ficheros .txt y se creaban las estructuras de datos necesarias (árbol de rutas, vuelos, wapoints, etc).
	\item finalmente se creaba un problema de CIPLEX, el cual buscaba la solución óptima del problema.
\end{enumerate}

\subsection{Nuevos objetivos}
Los objetivos marcados para este TFG son los siguientes (ordenados en decreciente prioridad):
\begin{enumerate}
	\item \textbf{Mejorar heurístico: } en el momento de iniciar el TFG, el heurístico existente era un algoritmo de Greedy, que no era demasiado eficiente.
	\item \textbf{Mejorar el sistema de lectura de datos:} el sistema actual crea ficheros .txt que pueden superar las 100.000 lineas. hay que mejorar este sistema.
	\item \textbf{representación gráfica: }aunque no es estrictamente necesaria, si se dispusiera de tiempo suficiente se añadiriía una representación gráfica de la solución del problema. 	 
\end{enumerate}